\documentclass[12pt]{article}
%\usepackage[html,1,sections+,pic-array,pic-eqnarray,pic-tabbing,pic-tabular,fonts+]{tex4ht}
\usepackage[T1]{fontenc}
%\usepackage[T2C]{fontenc}
\usepackage[full]{textcomp}
%\usepackage{calc}
\usepackage{array}
\usepackage{amsmath,amsfonts,amssymb,euscript}
%\usepackage[english,russian]{babel}
\usepackage[euler-digits]{eulervm}
\usepackage{color}
\usepackage{tabularx}
\usepackage{paralist}
\usepackage{xspace}
%\usepackage{fancyvrb}
%\usepackage{fancyhdr}

%\definecolor{darkgreen}{rgb}{0,0.5,0}
%\definecolor{darkblue}{rgb}{0,0,0.5}

\newcommand{\FORTRAN}{\textsf{FORTRAN}\xspace}

%\renewcommand\tabularxcolumn[1]{>{\small}m{#1}}
%\newcolumntype{Y}{>{\vphantom{$\big|$}\raggedright\arraybackslash}X}
%\newcolumntype{Y}{>{\raggedright\arraybackslash}X}


%\title{Runge Title}
%\author{Sergei Nikolaev}
%%%%%%%%%%%%%%%%%%%%%%%%%%%%%%%%%%%%%%%%%%%%%%%%%%%%%%%%%%%%%%%%%%%%%%%%
\begin{document}
%\maketitle

%\HPage{Runge Manual} 
%\EndHPage{}

% http://tug.org/applications/tex4ht/mn12.html#mn12-2

%\HPage{Introduction} 
\section{Introduction}
\subsection{What it does}
Runge is an Interactive Solver for Systems of 
Ordinary Differential Equations. It solves initial 
value problem (aka Cauchy problem) defined as the following:
for a given system of ordinary differential equations
\[
\dot x=F(t,x),\quad (t,x)\in \mathbb R^{n+1}
\]
and given initial values
\[
x(t_0)=x_0
\]
find solution 
\[
x(t_k)\in\mathbb R^n
\] 
at a given point of ``time'' i.e. for a given value \(t_k\) of \textit{independent variable} \(t\). 
Actually Runge produces solutions set
\[
x(t_0),x(t_1),x(t_2),\dots,x(t_k)
\]
where \(k\) is the number of steps taken. This allows to build 
\textit{trajectories} of solutions.


\subsection{Why Runge?}
\begin{compactitem}
\item It's fast. It utilizes BLAS and LAPACK \FORTRAN libraries optimized
for modern multi-core processors.
\item It's interactive. It allows you to start a solution by mouse click on a plane.
\item It's precise. It uses Runge Rule to adjust step length to satisfy required precision on each step.
\item It's effective. When it needs to compute derivatives (Jacobian matrix, for example) 
it does that \textit{analytically}, i.e. without using numerical methods.
\item It's portable. It works on Windows and Linux (32 and 64 bit versions) and Mac OSX (64 bit only).
\item It's open. It allows you to implement and embed your own algorithms (aka ``solvers'').
\item It's easy to use. It allows to export results to MS Excel and MATLAB.
\item It's free. It's distributed under 
\Link[http://www.boost.org/LICENSE_1_0.txt target="_blank"]{}{}Boost Software License\EndLink.
\end{compactitem}

\subsection{System Types and Solvers}
Runge comes with pre-installed solvers optimized for solving differential equations of different types:
\begin{compactitem}
\item Type 1. Generic non-autonomous system
\[
\dot x=F(t,x)
\]
\item Type 2. Generic autonomous system (it's a subset of Type 1, 
i.e. you can use both types 1 and 2 for autonomous sytems --- sometimes 
it makes sense for choosing appropriate solver)
\[
\dot x=F(x)
\]
\item Type 3. Pseudo-linear system 
(here it's assumed that \(\phi(x)\) is relatively small compared to matrix \(A(t)\))
\[
\dot x=A(t)x+\phi(x)
\]
\item Type 4. Pseudo-linear system with constant matrix B
\[
\dot x=Bx+f(t,x)
\]
\end{compactitem}
The solvers (algorithms) coming in standard package are:
\begin{compactitem}
\item Runge-Kutta process modification developed by R. England. 
Fast and precise method of fifth order suitable for solving systems of Type 1.
See \cite{England}.
\item Exponential method modification developed by J.D. Lawson. 
It's recommended for linear and quasi-linear systems of Type 1,3 and 4 (including stiff ones).
The method is A-stable for linear systems, i.e. residuals do not depend on step length.
See \cite{Lawson}.
\item Implicit process developed by H.H. Rosenbrock. 
It's recommended for non-linear systems of Type 1 and 2 (including stiff ones).
See \cite{Rosenbrock}.
\end{compactitem}

\subsection{Expressions and Functions}

\Link{}{expressions}\EndLink
The following functions and operators are supported for programming the systems mentioned above.


%\setlength\extrarowheight{2pt}
\begin{tabular}{|>{\large\ttfamily}r|l|} 
\hline
+ - * / ^ & arithmetic operators: add, subtract, multiply, divide, power\\\hline
exp(x) & \(e^x\)\\\hline
sqrt(x) & \(\sqrt x\)\\\hline
log(x) & natural logarithm of x\\\hline
log10(x) & common (base 10) logarithm of x\\\hline
sin(x) & sine of x\\\hline
cos(x) & cosine of x\\\hline
tan(x) & tangent of x\\\hline
asin(x) & arc sine of x\\\hline
acos(x) & arc cosine of x\\\hline
atan(x) & arc tangent of x\\\hline
sinh(x) & hyperbolic sine of x\\\hline
cosh(x) & hyperbolic cosine of x\\\hline
tanh(x) & hyperbolic tangent of x\\\hline
sinint(x) & sine integral of x \[\int_0^x \frac{\sin t}{t}dt\]\\\hline
cosint(x) & cosine integral of x \[-\int_x^\infty \frac{\cos t}{t}dt\]\\\hline
sign(x) & sign of x 
\[
 \left\{
  \begin{aligned}
-1 & \text{ if } x < 0\\
 0 & \text{ if } x = 0\\
 1 & \text{ if } x > 0
  \end{aligned}
 \right.
\]\\\hline
abs(x) & \(|x|\)\\\hline
iif(x,expr1,expr2) & immediate if
\[
 \left\{
  \begin{aligned}
expr1 & \text{ if } x < 0\\
expr2 & \text{ if } x \ge 0
  \end{aligned}
 \right.
\]\\\hline
%delta(x,y) & delta function of x and y
%\[
% \left\{
%  \begin{aligned}
%\infty & \text{ if } x = y\\
%0 & \text{ if } x \ne y
%  \end{aligned}
% \right.
%\]\\\hline
sat(x,y) & satellite function of x and y
\[
 \left\{
  \begin{aligned}
1 & \text{ if } x > |y|\\
0 & \text{ if } -|y| \le x \le |y|\\
-1 & \text{ if } x < -|y|
  \end{aligned}
 \right.
\]\\\hline
i & 1 (one)\\\hline
 & 0 (empty field means zero)\\\hline
\end{tabular}

Examples: \texttt{2*sin(t-1)+cos(t)-x^2}, \texttt{sqrt(abs(x))}, \texttt{iif(t,sin(x),cos(x))} etc.



\section{Quick Tour}

When started, Runge looks like this:

\Picture[Main Window]{mainWindow.png}

There are 4 areas here:
\begin{enumerate}
\item Menu and toolbar.
\item Left pane with four selectable sub-panes.
\item Solutions computed.
\item Right pane with equations you enter and solutions' values you compute.
\end{enumerate}

If you click Start button \Picture[Start Button]{startButton.png}
at this moment it would solve \(\dot x=0,\ t\in[0,1]\) problem.
That's because empty field for \(F(t,x)\) system on the right pane is equivalent to \(0\)
and default range for independent variable is \([0,1]\). Click Start button \Picture[Start Button]{startButton.png}
or choose ``Start'' item in ``Run'' menu

\Picture[Run Menu]{runMenu.png}

Select ``Solutions'' subpane and click on first solution to see results on the right pane:

\Picture[Solutions]{mainWindowSolutions.png}

The columns you see out there are: values of independent variables \(t_i, i=0,1,\dots\), 
solution values \(x(t_i), i=0,1,\dots\) and \textit{recommended} step values.
You can also delete this solution or export it to MS Excel comma-separated values file or to MATLAB script
using the following buttons:

\Picture[]{exportRemove.png}

Click 2D Draw button \Picture[2D Draw Button]{2DdrawButton.png} to see solution graph:

\Picture[2D Draw Window]{2DdrawWindow.png}



%%%%%%%%%%%%%%%%%%%%%%%%%%%%%%%%%%%%%%%%%%%%%%%%%%%%%%%%%%%%%%%%%%%%%%%%%%%%%%%%%%%%%%%%%%%%%%%%%%%%%%


\section{Main Menu}

\subsection{File Menu}

\Picture[]{fileMenu.png}

Here you can:
\begin{compactitem}
\item Start new file (new system to solve) by using Ctrl~N (Cmd~N) combination.
\item Open existing Runge file by using Ctrl~O (Cmd~O) combination.
\item Open recently used Runge file by using this sub-menu:

\Picture[]{fileMenuMRU.png}

\item Save current Runge file by using Ctrl~S (Cmd~S) combination.
\item Save current Runge file with different name.
\end{compactitem}

Runge file stores the system you entered and all solutions and their properties.


\subsection{Options Menu}

Windows and Linux: \Picture[]{optionsMenu.png} Mac: \Picture[]{optionsMenuMac.png}

Here you can:
\begin{compactitem}
\item Choose Runge configuration file. On Mac use main "Runge" menu, item "Preferences".
 Use it only if you want to add your own algorithms (aka ``solvers''). 
Keep default Runge*.xml configuration otherwise.
\item Choose font for entering expressions.
\item Choose language (restart required). Currently English and Russian are supported only.
\end{compactitem}


\subsection{Run Menu}

\Picture[]{runMenu.png}

Here you can:
\begin{compactitem}
\item Start solving your system.
\item Puse it if it's started.
\item Resume it if it's paused.
\item Stop it if it's started.
\end{compactitem}


\subsection{View Menu}


\Picture[]{graphicsMenu.png}

Call it to open 2D or 3D Drawer window.

\subsection{Help Menu}

Call it to open Runge Manual.



%%%%%%%%%%%%%%%%%%%%%%%%%%%%%%%%%%%%%%%%%%%%%%%%%%%%%%%%%%%%%%%%%%%%%%%%%%%%%%%%%%%%%%%%%%%%%%%%%%%%%%



\section{Programming and Solving}

Before solving your system of ordinary differential equations you need to program it in Runge.
This section describes the way you can do that.

\subsection{System}

\begin{enumerate}

\item Choose type of your system:

\Picture[System Type]{systemType.png}

\item Choose dimension:

\Picture[Dimension]{dimension.png}

\item Choose independent variable. It should be valid identifier (i.e. a string beginning with letter):

\Picture[Independent Variable]{indepVar.png}

\item Enter the system you'd like to solve. First column is for dependent variables 
(valid identifiers, i.e. strings beginning with letter), second one is for equations.
Equations should be valid \Link{expressions}{}expressions\EndLink{}
containing arithmetic operators and elementary functions.
You can also use macros (like macro ``p'' shown below) for repeating expressions:

\Picture[System]{systemKrug.png}

The following system is entered at this picture:
\[
 \left\{
  \begin{aligned}
\dot x&=y+xp\\
\dot y&=-x+yp
  \end{aligned}
 \right.
\]


\end{enumerate}


\subsection{Parameters}

\begin{enumerate}


\item Choose solver. Not every solver works for a given system type:

\Picture[Solver]{solver.png}

\item Set start parameters, values and macros:

\Picture[Parameters]{parametersWindow.png}

Here you can set:
 
\begin{compactitem}
\item \textbf{Start t} That's the initial value of the independent variable.
\item \textbf{End t} That's the end value of the independent variable --- last solution will be computed at this point.
\item \textbf{H} Initial step. Real step will be computed according to precision required.
\item \textbf{Hmin} Minimum step allowed. If precision can't be reached even for this step, solver fails.
\item \textbf{Hmax} Maximum step allowed.
\item \textbf{Eps} Absolute precision required on each step.
\item \textbf{P} Relative precision measure. If some component's absolute value becomes greater than P, 
it gets normalized when precision compared to Eps.
\item \textbf{Macros} List with macros. Here we have
\[
p=2-x^2-y^2
\]
therefore the final system turns to be 
\[
 \left\{
  \begin{aligned}
\dot x&=y+x(2-x^2-y^2)\\
\dot y&=-x+y(2-x^2-y^2)
  \end{aligned}
 \right.
\]
\item \textbf{Dep. var. - Start Value} Initial values for each dependent variable.
In this particular case we have:
\[
x(t_0)=2,\ y(t_0)=2,
\]
where \(t_0=0\), see Start t above.
\end{compactitem}

\end{enumerate}



\subsection{Solution}

\begin{enumerate}

\item Click Start button \Picture[Start Button]{startButton.png} to start solving the system.
For big systems or long time ranges you might see progress bar

\Picture[Progress Bar]{progressBar.png}

Here you can stop, pause and resume solving process by using appropriate buttons.

\item Explore solution

\Picture[]{krugSolution.png}

Every solution has an id starting from zero.

\item Explore solution graphically (here you can select different 
variables for X and Y axes):

\Picture[]{krugTX.png}
\Picture[]{krugXY.png}

\end{enumerate}

\subsection{Export}

Use Export feature to export solution to other programs like MS Excel or MATLAB

\Picture[]{exportRemove.png}

Excel should read it like this:

\Picture[]{toExcel.png}

MATLAB should read it like this:

\Picture[]{toMATLAB.png}

And after issuing a command like

\texttt{>{}> plot(solution0(:,2), solution0(:,3));}

it should render a plot:

\Picture[]{plotMATLAB.png}



%%%%%%%%%%%%%%%%%%%%%%%%%%%%%%%%%%%%%%%%%%%%%%%%%%%%%%%%%%%%%%%%%%%%%%%%%%%%%%%%%%%%%%%%%%%%%%%%%%%%%%





\section{2D Graphics}

\subsection{2D Drawer window}

Click \Picture[]{2DdrawButton.png} to bring 2D Drawer window which can also be used to start 
solving the system you entered interactively. You can open few windows by clicking this button multiple times.

\Picture[]{2DdrawParts.png}

\begin{enumerate}
\item Toolbar (each button function explained below).
\item X axis variable. This is independent variable by default.
\item Y axis variable.
\item Graphics plane
\item Y axis with marks. Mark step is 1 coordinate unit (adjustable), marks can be turned off.
\item X axis with marks. Mark step is 1 coordinate unit (adjustable), marks can be turned off.
\item Solution. It has a holder at the beginning looking like a small circle. Click on it to select a solution 
(color gets changed, \textit{solution also gets selected in main window}). Click somewhere else to unselect.
\item Current mouse coordinates (this is handy to start new solutions).
\item Resizing handle. You can resize this window as much as you want.
\end{enumerate}

\subsection{Interactively starting solution from given point}

Click ``Run froim point'' button \Picture[]{runFromPointButton.png} to enter the mode (mouse cursor will be changed accordingly).
Click on a plain where you want to run new solution from. You should get something like this:

\Picture[]{newSolution.png}

Note that new solution gets added to the main window list and it's also drawn on every other drawer open.

You can quickly create a set of solutions by clicking at different points:

\Picture[]{solutionsSet.png}



\subsection{Selecting solutions}

Every solution has a first point where independent variable \(t=t_0\).
It's marked by a small circle called ``handle'':

\Picture[]{unselectedSolutions.png}

Click it to select:

\Picture[]{selectedSolution.png}

Note how it gets selected in main window (it works in both ways --- you can select it in the list as well).
You can explore, export and remove selected solution.

Selected solution is marked by different color for each second step.
You choose the color using the color menu (since version 1.1, see below).
Sometimes it's handy to observe how steps change:

\Picture[]{selectedStripes.png}






\subsection{Zoom and Pan}

Click Zoom button \Picture[]{zoomButton.png} to bring zoom menu

\Picture[]{zoomMenu.png}

Select Zoom In or press Ctrl~+ combination to increase the drawing in 2:1 ratio:

\Picture[]{solutionsSetZoomed.png}

Zoom in level is virtually endless.
Use ``Zoom Out'' to unzoom (Ctrl~--). Use ``Zoom to rectangle'' (Ctrl~Z) to select a rectangle by mouse.
This rectangle would be scaled to. Use ``Fit all'' (Ctrl~F) to show everything.

Click Pan button \Picture[]{panButton.png} to enter Panning mode (mouse cursor changes).
You can drag your drawing for exploring its different parts. Panning is available only 
when there is something out of window borders.


\subsection{Color}

Click Palette button 
%\Picture[]{colorButton.png}
 to bring the color menu:

\Picture[]{colorMenu.png}

By selecting first item you can:

\begin{compactitem}
\item change color of selected solution (if one is selected)
\item change \textit{current} color for all subsequet solutions otherwise
\end{compactitem}

Please note that said above is also valid for all odd steps.

By selecting second item you can:

\begin{compactitem}
\item change color of each even step of selected solution (if one is selected)
\item change \textit{current} color of each even step for all subsequet solutions otherwise
\end{compactitem}

For example:

\Picture[]{coloredSolutions.png}


\subsection{Settings}

Click Drawer Settings button \Picture[]{drawerSettingsButton.png} to bring settings window:

\Picture[]{drawerSettings.png}

Here you can:
\begin{enumerate}
\item Turn on/off axes. If axes are turned on you can:
\begin{compactitem}
\item Click Adjust button to adjust axes lengths automatically according to current drawing.
\item Click Color... button to set axes color.
\item Edit semi-axes lengths manually (in coordinate units).
\end{compactitem}

\item Turn on/off line widths for axes and solutions. When turned off, the lines are drawn 
without widths at any scale level (``hair'' lines). Sometimes for better 
quality of exported images you'd need to
set those widths manually. 

\item Turn on/off axes marks. If axes marks are turned on you can:
\begin{compactitem}
\item Change X axis step (default is 1).
\item Change Y axis step (default is 1).
\item Set length of marks \textit{in pixels}.
\end{compactitem}

\item Change diameter of solution handles \textit{in pixels}.

\item Make the drawer to rescale each time new solution added. This mode (default) helps 
to keep seeing everything.
\end{enumerate}


\subsection{Export to File}


Click Export to File button \Picture[]{exportButton.png} to open Export window:

\Picture[]{exportToImage.png}

Here you can:
\begin{enumerate}
\item Choose image type:

\Picture[]{imageTypes.png}

This list is platform-dependent.
\item Set image width and height in pixels.
\item Set JPEG quality if it's selected.
\item Make the drawer to open newly created image using OS default viewer:

\Picture[]{exportedImage.png}

Here is 300x300 result sample is shown:

\Picture[]{exportedSolution.png}

\end{enumerate}


\subsection{Print to Document/Paper}

Click Print button \Picture[]{printButton.png} to open Print to Document/Paper window:

\Picture[]{printWindow.png}

\begin{enumerate}
\item Choose where to print: to PDF, PS or paper.
\item Choose document/paper size:

\Picture[]{documentSizes.png}

This list is platform-dependent.

\item Set different lines widths and handle diameter im millimeters.

\item Make the drawer to open newly created document using OS default viewer:

\Picture[]{acrobat.png}

\end{enumerate}


\subsection{Help}

Click Help button \Picture[]{helpButton.png} to open Runge Manual.



%%%%%%%%%%%%%%%%%%%%%%%%%%%%%%%%%%%%%%%%%%%%%%%%%%%%%%%%%%%%%%%%%%%%%%%%%%%%%%%%%%%%%%%%%%%%%%%%%%%%%%





\section{3D Graphics}

\subsection{3D Drawer window}

Click \Picture[]{3DdrawButton.png} to bring 3D Drawer window (available when system dimension is 2 or greater). 
You can open few windows by clicking this button multiple times.

\Picture[]{3DdrawWindow.png}

Here you can choose what variable to assign to what axis and 
rotate solutions drawing using slides and spinners on the left side. Left/right click + mouse drag
rotate by XY and XZ exes respectively. Use Zoom spinner to scale the drawing.


\subsection{Pan}

Click Pan button \Picture[]{panButton.png} to enter Panning mode (mouse cursor changes).
You can drag your drawing for exploring its different parts. 



\subsection{Settings}

Click Drawer Settings button \Picture[]{drawerSettingsButton.png} to bring settings window:

\Picture[]{3DdrawerSettings.png}

Here you can:
\begin{enumerate}
\item Turn on/off axes. If axes are turned on you can:
\begin{compactitem}
\item Edit axes' lengths manually (in coordinate units).
\item Click Color... button to set each axis color.
\item Click Adjust button to adjust axes lengths automatically according to current drawing.
\end{compactitem}
\item Eidt line widths for axes and solutions (in pixels). 
\item Change diameter of solution handles (in pixels).
\end{enumerate}



\subsection{Export to File}

Click Export to File button \Picture[]{exportButton.png} to open Export window:

\Picture[]{exportToImage.png}

Here you can:
\begin{enumerate}
\item Choose image type:

\Picture[]{imageTypes.png}

This list is platform-dependent.
\item Set image width and height in pixels.
\item Set JPEG quality if it's selected.
\item Make the drawer to open newly created image using OS default viewer.
\end{enumerate}


\subsection{Help}

Click Help button \Picture[]{helpButton.png} to open Runge Manual.


%%%%%%%%%%%%%%%%%%%%%%%%%%%%%%%%%%%%%%%%%%%%%%%%%%%%%%%%%%%%%%%%%%%%%%%%%%%%%%%%%%%%%%%%%%%%%%%%%%%%%%



\begin{thebibliography}{9}
\bibitem{England}
\textit{R. England.} Error Estimates for Runge-Kutta Type Solutions to Systems of 
Ordinary Differential Equations 
// Research and Development Department, Pressed Steel Fisher Ltd., Cowley, Oxford, UK. October 1968
\bibitem{Lawson}
\textit{Lawson J.D.} Generalized Runge-Kutta processes for stable systems with large Lipschitz constants 
// SIAM Journal on Numerical Analysis, 1967, V. 4, No 3.
\bibitem{Rosenbrock}
\textit{H.H. Rosenbrock} Some general implicit processes for the numerical solution of differential equations
Comput. J., 5 (1963) pp.329-330.
\end{thebibliography}

%Back to \ExitHPage{Introduction}
%\EndHPage{}


\end{document}
