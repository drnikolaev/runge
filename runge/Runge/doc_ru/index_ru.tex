\documentclass[12pt]{article}
%\usepackage[html,1,sections+,pic-array,pic-eqnarray,pic-tabbing,pic-tabular,fonts+]{tex4ht}
\usepackage[utf8]{inputenc} 
\usepackage[full]{textcomp}
%\usepackage{calc}
\usepackage{array}
\usepackage{amsmath,amsfonts,amssymb,euscript}
\usepackage[english,russian]{babel}
\usepackage[euler-digits]{eulervm}
\usepackage{color}
\usepackage{tabularx}
\usepackage{paralist}
\usepackage{xspace}
%\usepackage{fancyvrb}
%\usepackage{fancyhdr}

%\definecolor{darkgreen}{rgb}{0,0.5,0}
%\definecolor{darkblue}{rgb}{0,0,0.5}

\newcommand{\FORTRAN}{\textsf{ФОРТРАН}\xspace}

%\renewcommand\tabularxcolumn[1]{>{\small}m{#1}}
%\newcolumntype{Y}{>{\vphantom{$\big|$}\raggedright\arraybackslash}X}
%\newcolumntype{Y}{>{\raggedright\arraybackslash}X}


%\title{Runge Title}
%\author{Sergei Nikolaev}
%%%%%%%%%%%%%%%%%%%%%%%%%%%%%%%%%%%%%%%%%%%%%%%%%%%%%%%%%%%%%%%%%%%%%%%%
\begin{document}
%\maketitle
\selectlanguage{russian}

\section{Введение}
\subsection{Что делает Рунге}
Рунге -- это интерактивный решатель систем 
обыкновенных дифференциальных уравнений (ОДУ).
Он решает задачи с начальными значениями (т.н. задачи Коши),
которые могут быть определены следующим образом:
для данной системы обыкновенных дифференциальных уравнений
\[
\dot x=F(t,x),\quad (t,x)\in \mathbb R^{n+1}
\]
и данных начальных значений
\[
x(t_0)=x_0
\]
найти решение
\[
x(t_k)\in\mathbb R^n
\]
для данного значения ``времени'',
т.е. для данного значения \(t_k\) \textit{независимой переменной} \(t\). 
Рунге также вычисляет множество решений
\[
x(t_0),x(t_1),x(t_2),\dots,x(t_k)
\]
где \(k\) -- количество выполненных шагов. Это позволяет строить 
\textit{траектории} решений.


\subsection{Почему Рунге?}
\begin{compactitem}
\item Он быстрый. Рунге использует \FORTRAN-библиотеки BLAS и LAPACK, 
оптимизированные для современных многоядерных процессоров.
\item Он интерактивный. Рунге позволяет начинать решение щелчком мыши из точки на плоскости.
\item Он точный. Рунге использует правило Рунге для выбора оптимальной длины шага, обеспечивающей заданную точность.
\item Он эффективный. В том случае, когда нужно вычислять производные 
(например, элементы матрицы Якоби),
Рунге делает это \textit{аналитически}, т.е. не прибегая к разностным схемам.
\item Он переносимый. Рунге работает на платформах Windows и Linux (32-х и 64-х разрядные версии) и Mac OSX (64-х разрядная версия).
\item Он открытый. Рунге позволяет добавлять ваши собственные алгоритмы.
\item Он лёгок в использовании. Рунге позволяет экспортировать решения в программы MS Excel и MATLAB.
\item Он бесплатный. Рунге распространяется по свободной лицензии 
\Link[http://www.boost.org/LICENSE_1_0.txt target="_blank"]{}{}Boost Software License\EndLink.
\end{compactitem}

\subsection{Типы систем и алгоритмы}
Рунге поставляется с предустановленными алгоритмами, оптимизированными для решения систем следующих типов:
\begin{compactitem}
\item Тип 1. Неавтономная система общего вида
\[
\dot x=F(t,x)
\]
\item Тип 2. Автономная система общего вида (этот тип является подмножеством Типа 1,
т.е. для автономных систем возможно использование обоих типов, что порой имеет смысл
для более широкого выбора алгоритмов)
\[
\dot x=F(x)
\]
\item Тип 3. Псевдо-линейная система 
(здесь предполагается, что \(\phi(x)\) относительно мало по сравнению с \(A(t)\))
\[
\dot x=A(t)x+\phi(x)
\]
\item Тип 4. Псевдо-линейная система с постоянной матрицей B
\[
\dot x=Bx+f(t,x)
\]
\end{compactitem}
Следующие алгоритмы поставляются с программой:
\begin{compactitem}
\item Runge-Kutta process modification developed by R. England. 
Быстрый и точный алгоритм 5-го порядка для решения систем 1-го типа.
См. \cite{England}.
\item Exponential method modification developed by J.D. Lawson. 
Рекомендуется для линейных и псевдо-линейных систем типов 1, 3 и 4 
(включая жёсткие системы).
Этот алгоритм А-стабилен для линейных систем, 
т.е. погрешности не зависят от длины шага.
См. \cite{Lawson}.
\item Implicit process developed by H.H. Rosenbrock. 
Рекомендуется для нелинейных систем типов 1 и 2 (включая жёсткие системы).
См. \cite{Rosenbrock}.
\end{compactitem}

\subsection{Выражения и функции}

\Link{}{expressions}\EndLink
Следующие функции и операторы поддерживаются для программирования систем, рассмотренных выше.

%\setlength\extrarowheight{2pt}
\begin{tabular}{|>{\large\ttfamily}r|l|} 
\hline
+ - * / ^ & арифметические операторы: сумма, разность умножение, деление, степень\\\hline
exp(x) & \(e^x\)\\\hline
sqrt(x) & \(\sqrt x\)\\\hline
log(x) & натуральный логарифм x\\\hline
log10(x) & логарифм x по основанию 10\\\hline
sin(x) & синус x\\\hline
cos(x) & косинус x\\\hline
tan(x) & тангенс x\\\hline
asin(x) & арксинус x\\\hline
acos(x) & арккосинус x\\\hline
atan(x) & арктангенс x\\\hline
sinh(x) & гиперболический синус x\\\hline
cosh(x) & гиперболический косинус x\\\hline
tanh(x) & гиперболический тангенс x\\\hline
sinint(x) & интегральный синус x \[\int_0^x \frac{\sin t}{t}dt\]\\\hline
cosint(x) & интегральный косинус x \[-\int_x^\infty \frac{\cos t}{t}dt\]\\\hline
sign(x) & знак x 
\[
 \left\{
  \begin{aligned}
-1 & \text{ if } x < 0\\
 0 & \text{ if } x = 0\\
 1 & \text{ if } x > 0
  \end{aligned}
 \right.
\]\\\hline
abs(x) & \(|x|\)\\\hline
iif(x,expr1,expr2) & условный оператор
\[
 \left\{
  \begin{aligned}
expr1 & \text{ if } x < 0\\
expr2 & \text{ if } x \ge 0
  \end{aligned}
 \right.
\]\\\hline
%delta(x,y) & delta function of x and y
%\[
% \left\{
%  \begin{aligned}
%\infty & \text{ if } x = y\\
%0 & \text{ if } x \ne y
%  \end{aligned}
% \right.
%\]\\\hline
sat(x,y) & сателлит-функция x и y
\[
 \left\{
  \begin{aligned}
1 & \text{ if } x > |y|\\
0 & \text{ if } -|y| \le x \le |y|\\
-1 & \text{ if } x < -|y|
  \end{aligned}
 \right.
\]\\\hline
i & 1 (единица)\\\hline
 & 0 (пустое поле означает ноль)\\\hline
\end{tabular}

Примеры: \texttt{2*sin(t-1)+cos(t)-x^2}, \texttt{sqrt(abs(x))}, \texttt{iif(t,sin(x),cos(x))} и т.д.



\section{Краткий обзор}

После запуска Рунге выглядит так:

\Picture[Main Window]{mainWindow.png}

Главное окно имеет четыре основных части:
\begin{enumerate}
\item Меню и панель инструментов.
\item Левая панель с четырьмя закладками (см. ниже).
\item Список вычисленных решений.
\item Правая панель (здесь вводится система и показываются решения).
\end{enumerate}

Если нажать кнопку Старт \Picture[Start Button]{startButton.png},
то будет решена задача \(\dot x=0,\ t\in[0,1]\),
поскольку пустое поле для правой части системы \(F(t,x)\) 
на правой панели эквивалентно \(0\),
а интервал для независимой переменной по умолчанию равен \([0,1]\). 
Нажмите кнопку Старт \Picture[Start Button]{startButton.png}
или выберите пункт ``Старт'' в меню ``Старт''

\Picture[]{runMenu.png}

Выберите закладку ``Решения'' 
и щёлкните на решении, чтобы увидеть результаты на правой панели:

\Picture[Solutions]{mainWindowSolutions.png}

Колонки справа соответствуют: значениям независимой переменной \(t_i, i=0,1,\dots\), 
значениям решения \(x(t_i), i=0,1,\dots\) и \textit{рекомендуемым} размерам шага.
Вы также можете удалить это решение или экспортировать его в MS Excel или MATLAB файл используя следующее меню:

\Picture[]{exportRemove.png}

Нажмите кнопку ``2D График'' \Picture[2D Draw Button]{2DdrawButton.png}, 
чтобы увидеть график решения:

\Picture[2D Draw Window]{2DdrawWindow.png}



%%%%%%%%%%%%%%%%%%%%%%%%%%%%%%%%%%%%%%%%%%%%%%%%%%%%%%%%%%%%%%%%%%%%%%%%%%%%%%%%%%%%%%%%%%%%%%%%%%%%%%


\section{Главное меню}

\subsection{Меню Файл}

\Picture[]{fileMenu.png}

Это меню позволяет:
\begin{compactitem}
\item Создать новый файл (новую систему ОДУ для решения) 
используя комбинацию Ctrl~N.
\item Открыть существующий файл Рунге используя комбинацию Ctrl~O.
\item Открыть ранее сохранённый файл при помощи под-меню

\Picture[]{fileMenuMRU.png}

\item Сохранить текущий файл используя комбинацию Ctrl~S.
\item Сохранить текущий файл под другим именем.
\item Выйти из Рунге.
\end{compactitem}

Файл Рунге хранит систему, все параметры, все решения и их свойства.

\subsection{Меню Настройки}

\Picture[]{optionsMenu.png}

Это меню позволяет:
\begin{compactitem}
\item Выбрать конфигурационный файл Рунге. Используйте это только в том случае,
когда требуется добавить свой собственный алгоритм.
Используйте стандартный файл Runge*.xml в противном случае.
\item Выбрать шрифт для программирования выражений 
(моноширинные шрифты семейства Courier улучшают читаемость).
\item Выберите язык (требуется перезагрузка программы в случае смены языка). 
В настоящее время поддерживаются русский и английский языки. 
\end{compactitem}


\subsection{Меню Старт}

\Picture[]{runMenu.png}

Это меню позволяет:
\begin{compactitem}
\item Начать решение системы.
\item Приостановить, если решение начато.
\item Продолжить, если решеие приостановлено.
\item Остановить, если решение начато.
\end{compactitem}


\subsection{Меню График}

\Picture[]{graphicsMenu.png}

Открывает окно с 2D или 3D графиками решений (можно открыть несколько окон).

\subsection{Меню Справка}

Открывает справочное руководство по Рунге.


%%%%%%%%%%%%%%%%%%%%%%%%%%%%%%%%%%%%%%%%%%%%%%%%%%%%%%%%%%%%%%%%%%%%%%%%%%%%%%%%%%%%%%%%%%%%%%%%%%%%%%



\section{Программирование и решение систем}

Прежде чем решить систему ОДУ, ее нужно запрограммировать в Рунге.
Эта глава объясняет, как это делать.

\subsection{Система}

\begin{enumerate}

\item Выберите тип системы:

\Picture[System Type]{systemType.png}

\item Выберите размерность:

\Picture[Dimension]{dimension.png}

\item Выберите независимую переменную. Это должен быть идентификатор (т.е. строка, начинающаяся с буквы):

\Picture[Independent Variable]{indepVar.png}

\item Введите систему, которую нужно решить. 
Левая колонка для имен зависимых переменных
(идентификаторы, т.е. строки, начинающиеся с буквы), 
правая колонка для правых частей уравнений системы.
Уравнения должны быть корректными \Link{expressions}{}выражениями\EndLink{},
содержащими арифметические операторы и элементарные функции.
Возможно также использование шаблонов (как шаблон ``p'' ниже) 
для повторяющихся выражений:

\Picture[System]{systemKrug.png}

Здесь введена система
\[
 \left\{
  \begin{aligned}
\dot x&=y+xp\\
\dot y&=-x+yp
  \end{aligned}
 \right.
\]
\end{enumerate}


\subsection{Параметры}

\begin{enumerate}

\item Выберите алгоритм. Не каждый алгоритм подходит для выбранного типа системы:

\Picture[Solver]{solver.png}

\item Установите параметры, начальные значения и шаблоны (если они используются):

\Picture[Parameters]{parametersWindow.png}

Здесь задаются:
 
\begin{compactitem}
\item \textbf{Start t} Начальное значение независимой переменной.
\item \textbf{End t} Конечное значение независимой переменной --- 
в этой точке будет вычислено последнее решение.
\item \textbf{H} Начальный шаг. В процессе решения этот шаг может быть 
изменен в соответствии с требуемой точностью.
\item \textbf{Hmin} Минимальный допустимый шаг. Если требуемая точность 
не может быть достигнута при этом шаге, 
Рунге прекращает решение.
\item \textbf{Hmax} Максимальный допустимый шаг.
\item \textbf{Eps} Абсолютная точность, требуемая на каждом шаге.
\item \textbf{P} Мера относительной погрешности. Если какая-либо компонента решения превышает этот порог,
она нормализуется перед сравнением с Eps.
\item \textbf{Шаблоны} Список шаблонов. Здесь мы имеем:
\[
p=2-x^2-y^2,
\]
таким образом, решаемая система будет иметь вид
\[
 \left\{
  \begin{aligned}
\dot x&=y+x(2-x^2-y^2)\\
\dot y&=-x+y(2-x^2-y^2)
  \end{aligned}
 \right.
\]
\item \textbf{Зависим. перем. - Начальное значение} 
Начальные значения для зависимых переменных.
В данном случае имеем:
\[
x(t_0)=2,\ y(t_0)=2,
\]
где \(t_0=0\), см. Start t выше.
\end{compactitem}

\end{enumerate}



\subsection{Решение}
\begin{enumerate}
\item Нажмите кнопку Старт \Picture[Start Button]{startButton.png} 
для запуска решения.
Для систем с большой размерностью или длинным интервалом независимой 
переменной возможно появление процентной линейки:

\Picture[Progress Bar]{progressBar.png}

Здесь можно остановить решение, сделать паузу и возобновить решение после паузы
используя соответствующие кнопки.

\item Исследуйте решение

\Picture[]{krugSolution.png}

Каждое решение имеет идентификатор, начинающийся с нуля.

\item Исследуйте решение графически (здесь можно выбрать разные переменные для осей X и Y):

\Picture[]{krugTX.png}
\Picture[]{krugXY.png}

\end{enumerate}

\subsection{Экспорт}
Используйте эту возможность для экспорта решений в программы MS Excel, MATLAB и им подобные:

\Picture[]{exportRemove.png}

Excel читает этот файл примерно так:

\Picture[]{toExcel.png}

MATLAB читает этот файл примерно так:

\Picture[]{toMATLAB.png}

И после запуска такой команды в MATLAB:

\texttt{>{}> plot(solution0(:,2), solution0(:,3));}

он должен нарисовать примерно следующее:

\Picture[]{plotMATLAB.png}



%%%%%%%%%%%%%%%%%%%%%%%%%%%%%%%%%%%%%%%%%%%%%%%%%%%%%%%%%%%%%%%%%%%%%%%%%%%%%%%%%%%%%%%%%%%%%%%%%%%%%%


\section{2D Графика}

\subsection{Графическое окно}

Нажмите \Picture[]{2DdrawButton.png}, чтобы открыть графическое окно.
Помимо изображения двумерного графика это окно может быть 
использовано для запуска решения системы
в интерактивном режиме. Возможно открытие нескольких окон одновременно.

\Picture[]{2DdrawParts.png}

\begin{enumerate}
\item Панель инструментов (каждая кнопка объяснена ниже).
\item Переменная для оси X. По умолчанию это независимая переменная.
\item Переменная для оси Y.
\item Графическая плоскость.
\item Ось Y с метками. Шаг меток равен одной координатной единице (это может быть настроено). 
Кроме того, метки могут быть отключены.
\item Ось X с метками. Шаг меток равен одной координатной единице (это может быть настроено). 
Кроме того, метки могут быть отключены.
\item Траектория решения. Она имеет маркер в виде маленького круга.
Щелкните по нему для выбора решения (цвет будет изменен, причем \textit{соответствующее решение
будет отмечено в главном окне}). Щелкните в другом месте для отмены выбора решения.
\item Текущие координаты указателя мыши (может быть полезно для начала нового решения).
\item Маркер для изменения размера окна.
\end{enumerate}

\subsection{Запуск решения в интерактивном режиме}
Нажмите кнопку ``Запустить расчет из точки'' \Picture[]{runFromPointButton.png} 
для входа в режим интерактивного решения (курсор мыши будет изменен).
Щелкните на плоскости по точке, из которой нужно начать решение.
Должно получиться следующее:

\Picture[]{newSolution.png}

Заметьте, что новое решение добавлено в главное окно, а также оно появляется на каждом графическом окне.
Таким образом можно быстро создать набор решений (фазовое поле), запуская их из разных точек:

\Picture[]{solutionsSet.png}


\subsection{Отметка решения}
Каждое решение имеет первую точку, где независимая переменная \(t=t_0\).
Эта точка отмечена маленьким кругом, так называемым ``маркером'':

\Picture[]{unselectedSolutions.png}

Щёлкните по нему, чтобы отметить решение:

\Picture[]{selectedSolution.png}

Заметьте, что решение отмечается и в списке на главном окне 
(это работает в обе стороны).
Отмеченное решение может быть просмотрено, экспортировано или удалено.
Отмеченное решение изображается другим цветом на каждом втором шаге.
Вы можете изменить этот цвет при помощи меню ``Цвет'' (начиная с версии 1.1, см. ниже).
Это бывает полезно для наблюдения за тем, как меняется длина шага:

\Picture[]{selectedStripes.png}


\subsection{Масштабирование и панорамирование}
Нажмите кнопку ``Масштаб'' \Picture[]{zoomButton.png} для открытия меню

\Picture[]{zoomMenu.png}

Выберите пункт ``Крупнее'' или нажмите Ctrl~+ для увеличение графика в масштабе 2:1:

\Picture[]{solutionsSetZoomed.png}

Уровень увеличения практически бесконечен.
Выберите пункт ``Мельче'' для уменьшения (комбинация Ctrl~--).
Выберите ``Увеличить до заданного прямоугольника'' (Ctrl~Z), чтобы отметить прямоугольник мышью
Этот прямоугольник будет увеличен на все окно.
Выберите ``Поместить все'' (Ctrl~F), чтобы поместить все решения в окне.

Нажмите кнопку ``Панорамирование'' \Picture[]{panButton.png} 
для входа в этот режим (курсор мыши изменится).
В этом режиме можно двигать график для просмотра его различных частей. 
Данный режим доступен только если есть хоть что-нибудь за пределами окна.


\subsection{Цвет}
Нажмите кнопку ``Цвет'' \Picture[]{colorButton.png} для того,
чтобы открыть меню ``Цвет'':

\Picture[]{colorMenu.png}

Выбрав первый пункт меню вы можете:

\begin{compactitem}
\item измененить цвет отмеченного решения (если одно решение отмечено)
\item измененить \textit{цвет по умолчанию} для всех последующих решений
\end{compactitem}

Этот пункт работает в том числе и для каждого нечётного шага.

Выбрав второй пункт меню вы можете:

\begin{compactitem}
\item измененить цвет каждого чётного шага отмеченного решения (если одно решение отмечено)
\item измененить \textit{цвет по умолчанию} каждого чётного шага для всех последующих решений
\end{compactitem}

Пример:

\Picture[]{coloredSolutions.png}


\subsection{Свойства графика}

Нажмите кнопку ``Свойства графика'' \Picture[]{drawerSettingsButton.png}, чтобы открыть окно свойств:

\Picture[]{drawerSettings.png}

Здесь можно:
\begin{enumerate}
\item Включить/выключить изображение осей. Если оси включены, то можно:
\begin{compactitem}
\item Нажать кнопку ``Установить'' для автоматического выравнивания длин осей под текущих график.
\item Нажать кнопку ``Цвет...'' для изменения цвета осей.
\item Редактировать длины полуосей (в координатных единицах) вручную.
\end{compactitem}

\item Включить/выключить ширину линий для осей и решений. 
Если ширина выключена, то ли нии рисуются одинаково тонко на 
любом устройстве при любом масштабе.
Иногда для лучшего качества изображений имеет смысл установить ширину вручную.

\item Включить/выключить метки осей. Если метки включены, то можно:
\begin{compactitem}
\item Изменить шаг меток по оси X (по умолчанию 1).
\item Изменить шаг меток по оси Y (по умолчанию 1).
\item Установить длину меток \textit{в пикселах}.
\end{compactitem}

\item Установить диаметр маркеров решений \textit{в пикселах}.

\item Установить авто-масштабирование графика сякий раз, 
когда добавляется новое решение.
Этот режим позволяет видеть всё. По умолчанию он включен.
\end{enumerate}


\subsection{Экспорт графика в файл}

Нажмите кнопку ``Экспорт графика в файл'' \Picture[]{exportButton.png} для открытия окна экспорта:

\Picture[]{exportToImage.png}

Здесь можно:
\begin{enumerate}
\item Выбрать тип изображения:

\Picture[]{imageTypes.png}

Этот список зависит от платформы.
\item Задать ширину и высоту будущего изображения в пикселах.
\item Установить качество кодирования JPEG.
\item Открывать созданное изображение во внешней программе, 
которая установлена по умолчанию на компьютере:

\Picture[]{exportedImage.png}

Здесь показан пример изображения размером 300х300 пиксел:

\Picture[]{exportedSolution.png}

\end{enumerate}


\subsection{Печать в документ или на бумагу}

Нажмите кнопку ``Печать'' \Picture[]{printButton.png} для открытия окна печати:

\Picture[]{printWindow.png}

\begin{enumerate}
\item Выберите, куда печатать: в PDF, PS или на бумагу.
\item Выберите размер документа/бумаги:

\Picture[]{documentSizes.png}

Этот список зависит от платформы.

\item Установите ширину линий и диаметр маркеров в миллиметрах.

\item Установите режим открывания нового документа во внешней программе, например:

\Picture[]{acrobat.png}

\end{enumerate}


\subsection{Справка}

Нажмите кнопку ``Справка'' \Picture[]{helpButton.png}, чтобы открыть данное руководство.


%%%%%%%%%%%%%%%%%%%%%%%%%%%%%%%%%%%%%%%%%%%%%%%%%%%%%%%%%%%%%%%%%%%%%%%%%%%%%%%%%%%%%%%%%%%%%%%%%%%%%%


\section{3D Графика}

\subsection{Графическое окно}

Нажмите \Picture[]{3DdrawButton.png}, чтобы открыть графическое окно (доступно только для систем размерности 2 и выше).
Вы можете открыть несколько окон нажатием той же кнопки.


\Picture[]{3DdrawWindow.png}

Здесь вы можете выбрать переменные для координатных осей,
а так же вращать график при помощи движков и полей ввода на левой стороне. 
При нажатой левой/правой кнопке мыши вращение происходит по осям XY и XZ соответственно.
Используйте поле Zoom для изменения масштаба.


\subsection{Панорамирование}

Нажмите кнопку  \Picture[]{panButton.png} 
для входа в этот режим (курсор мыши изменится).
В этом режиме можно двигать график для просмотра его различных частей. 

\subsection{Свойства графика}

Нажмите кнопку \Picture[]{drawerSettingsButton.png}, чтобы открыть окно свойств:

\Picture[]{3DdrawerSettings.png}

Здесь можно:
\begin{enumerate}
\item Включить/выключить изображение осей. Если оси включены, то можно:
\begin{compactitem}
\item Редактировать длины осей (в координатных единицах) вручную.
\item Нажать кнопку ``Цвет...'' для изменения цвета каждой из осей.
\item Нажать кнопку ``Установить'' для автоматического выравнивания длин осей под текущих график.
\end{compactitem}
\item Установить ширину линий для осей и решений. 
Иногда для лучшего качества изображений имеет смысл установить ширину вручную.
\item Установить диаметр маркеров решений (в пикселах).
\end{enumerate}


\subsection{Экспорт графика в файл}

Нажмите кнопку  \Picture[]{exportButton.png} для открытия окна экспорта:

\Picture[]{exportToImage.png}

Здесь можно:
\begin{enumerate}
\item Выбрать тип изображения:

\Picture[]{imageTypes.png}

Этот список зависит от платформы.
\item Задать ширину и высоту будущего изображения в пикселах.
\item Установить качество кодирования JPEG.
\item Открывать созданное изображение во внешней программе, 
которая установлена по умолчанию на компьютере.
\end{enumerate}


\subsection{Справка}

Нажмите кнопку \Picture[]{helpButton.png}, чтобы открыть данное руководство.


%%%%%%%%%%%%%%%%%%%%%%%%%%%%%%%%%%%%%%%%%%%%%%%%%%%%%%%%%%%%%%%%%%%%%%%%

\begin{thebibliography}{9}
\bibitem{England}
\textit{R. England.} Error Estimates for Runge-Kutta Type Solutions to Systems of 
Ordinary Differential Equations 
// Research and Development Department, Pressed Steel Fisher Ltd., Cowley, Oxford, UK. October 1968
\bibitem{Lawson}
\textit{Lawson J.D.} Generalized Runge-Kutta processes for stable systems with large Lipschitz constants 
// SIAM Journal on Numerical Analysis, 1967, V. 4, No 3.
\bibitem{Rosenbrock}
\textit{H.H. Rosenbrock} Some general implicit processes for the numerical solution of differential equations
Comput. J., 5 (1963) pp.329-330.
\end{thebibliography}

\end{document}
